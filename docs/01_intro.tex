\section{Introducción}

El problema clásico de interpolación Lagrangiana consiste en encontrar un polinomio $p \in \mathbb{P}^n([0,1])$ de grado a lo sumo $n$ tal que:
\begin{equation}\label{eq:interpolation}
p(x_j) = f_j, \quad j = 0, 1, \ldots, n
\end{equation}
donde $\{x_j\}_{j=0}^n \subseteq [0,1]$ son nodos de interpolación distintos y $\{f_j\}_{j=0}^n$ son datos dados.

La solución única existe y puede representarse en varias bases. Elegir la \textbf{base de Bernstein} es ventajoso en aplicaciones de geometría computacional (CAGD), métodos de elementos finitos de alto orden y teoría de aproximación de \textit{splines}.

\textbf{El Problema de Interpolación de Bernstein-Bézier:} Dado $\{x_j, f_j\}$, calcular los \textbf{puntos de control de Bézier} $\{c_k\}_{k=0}^n$ tales que el polinomio de Bernstein-Bézier
\begin{equation}\label{eq:bb_interp}
p(x) = \sum_{k=0}^n c_k B_k^n(x)
\end{equation}
satisfaga la condición (1.1), donde $B_k^n(x) = \binom{n}{k}(1-x)^{n-k}x^k$ son los polinomios base de Bernstein.

\textbf{Desafío Numérico:} El sistema lineal asociado tiene matriz de Bernstein-Vandermonde:
\begin{equation}
A_{ij} = B_i^n(x_j) = \binom{n}{i}(1-x_j)^{n-i}x_j^i
\end{equation}
Esta matriz está \textbf{altamente mal condicionada} (típicamente $\kappa(A) \sim 10^6$ a $10^{13}$).

\textbf{Contribuciones Previas:} Marco y Martínez (2007) desarrollaron un algoritmo $O(n^2)$ usando eliminación de Neville y propiedades de positividad total, pero la derivación es técnica y el algoritmo se limita al caso univariado.

\textbf{La Solución de Ainsworth-Sánchez:} Se presenta aquí un algoritmo alternativo con:
\begin{itemize}
\item \textbf{Misma complejidad} $O(n^2)$
\item \textbf{Derivación elemental} usando solo interpolación de Lagrange básica
\item \textbf{Estabilidad comparable} a Marco-Martínez
\item \textbf{Generalización natural} a dimensiones arbitrarias (contribución de M.A. Sánchez)
\end{itemize}
