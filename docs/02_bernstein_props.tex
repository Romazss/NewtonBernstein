\section{Propiedades Fundamentales de Bernstein}

Para desarrollar el algoritmo, necesitamos tres propiedades clave de los polinomios de Bernstein:

\subsection{P1: Relaciones de Producto}

Cualquier polinomio de grado 1 se puede escribir como:
\begin{equation}
x - x_0 = (1-x_0)B_1^1(x) - x_0 B_0^1(x)
\end{equation}
donde $B_1^1(x) = x$ y $B_0^1(x) = 1-x$.

Al multiplicar un polinomio de Bernstein de grado $n$ por esta expresión, obtenemos:
\begin{align}
B_1^1(x) \cdot B_k^n(x) &= \frac{k+1}{n+1} B_{k+1}^{n+1}(x) \label{eq:product1} \\
B_0^1(x) \cdot B_k^n(x) &= \frac{n+1-k}{n+1} B_k^{n+1}(x) \label{eq:product2}
\end{align}

\textit{Interpretación:} El producto de dos polinomios de Bernstein produce otro polinomio de Bernstein de grado incrementado.

\subsection{P2: Elevación de Grado}

Todo polinomio de grado $n-1$ puede expresarse en la base de Bernstein de grado $n$:
\begin{equation}\label{eq:degree_elevation}
B_k^{n-1}(x) = \frac{n-k}{n}B_k^n(x) + \frac{k+1}{n}B_{k+1}^n(x)
\end{equation}

\textit{Importancia:} Esta identidad permite pasar de grado $n-1$ a grado $n$ sin pérdida de información. Si $p_{n-1}(x) = \sum_{k=0}^{n-1} c_k^{(n-1)} B_k^{n-1}(x)$, entonces:
\begin{equation}\label{eq:elevation_coeffs}
p_{n-1}(x) = \sum_{k=0}^n \left(\frac{k}{n}c_{k-1}^{(n-1)} + \frac{n-k}{n}c_k^{(n-1)}\right) B_k^n(x)
\end{equation}
con la convención $c_{-1}^{(n-1)} = c_n^{(n-1)} = 0$.

\subsection{P3: Fórmula de Lagrange Mejorada}

El interpolante único en forma de Lagrange es:
\begin{equation}\label{eq:lagrange_form}
p(x) = \sum_{j=0}^n \mu_j f_j \frac{\ell(x)}{x - x_j}
\end{equation}
donde:
\begin{align}
\ell(x) &= \prod_{k=0}^n (x - x_k) \\
\mu_j &= \frac{1}{\ell'(x_j)}
\end{align}

Esta forma es numéricamente estable cuando se usa con aritmética de precisión finita \cite{Berrut2004}.
