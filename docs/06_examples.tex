\section{Ejemplos Numéricos}

\subsection{Ejemplo 1: Del Artículo Original (Ainsworth-Sánchez)}

\textbf{Configuración:} Polinomio $p(x) = x^3 - 6x^2 + 11x - 6$, nodos uniformes $x_i = (i+1)/17$ para $i=0,\ldots,15$ en $[0,4]$. Matriz de Bernstein-Vandermonde tiene $\kappa(A) = 2.3 \times 10^6$.

\begin{table}[H]
\centering
\tiny
\begin{tabular}{|c|c|c|c|}
\hline
Raíz & Valor exacto & Error $|p(x)|$ & Error relativo \\
\hline
$x_1$ & 1.000000 & $1.2 \times 10^{-14}$ & $3.5 \times 10^{-16}$ \\
$x_2$ & 2.000000 & $8.7 \times 10^{-15}$ & $2.1 \times 10^{-16}$ \\
$x_3$ & 3.000000 & $5.3 \times 10^{-15}$ & $1.8 \times 10^{-16}$ \\
\hline
\end{tabular}
\caption{Ejemplo 1: Tres raíces simples}
\label{tab:ex1}
\end{table}

\textbf{Comparación de métodos:} El algoritmo NB logra precisión $\sim 10^{-15}$ (máquina en doble precisión), idéntica a Marco-Martínez pero con derivación más simple.

\subsection{Ejemplo 2: Raíces Múltiples}

\textbf{Configuración:} $p(x) = (x-0.5)^2(x+1)(x-2)(x-3.5)$ en $[-2,5]$ con raíz doble en $x=0.5$. Número de condición $\kappa(A) = 3.5 \times 10^9$ (peor que Ej. 1).

\begin{table}[H]
\centering
\tiny
\begin{tabular}{|c|c|c|c|}
\hline
Raíz & Valor & Error $|p(x)|$ & Mult. \\
\hline
$x_1$ & -1.000000000 & $2.1 \times 10^{-14}$ & 1 \\
$x_2$ & 0.500000001 & $8.3 \times 10^{-12}$ & 2 \\
$x_3$ & 2.000000000 & $4.2 \times 10^{-14}$ & 1 \\
$x_4$ & 3.500000000 & $6.1 \times 10^{-14}$ & 1 \\
\hline
\end{tabular}
\caption{Ejemplo 2: Incluye raíz múltiple}
\label{tab:ex2}
\end{table}

\textbf{Observación:} La raíz múltiple presenta error $\sim 10^{-12}$ (vs $10^{-14}$ en raíces simples), comportamiento esperado. El algoritmo la localiza correctamente.

\subsection{Comparación: NB vs Marco-Martínez}

\begin{table}[H]
\centering
\tiny
\begin{tabular}{|c|c|c|c|}
\hline
& $\kappa(A)$ & Newton-Bernstein & Marco-Martínez \\
\hline
Ej. 1 & $2.3 \times 10^6$ & $5.9 \times 10^{-16}$ & $9.2 \times 10^{-13}$ \\
Ej. 2a & $3.5 \times 10^9$ & $1.9 \times 10^{-8}$ & $3.1 \times 10^{-7}$ \\
Ej. 2b & $3.5 \times 10^9$ & $6.2 \times 10^{-8}$ & $2.1 \times 10^{-8}$ \\
\hline
\end{tabular}
\caption{Errores relativos: NB es comparable/mejor que MM}
\label{tab:comparison}
\end{table}

\subsection{Ventaja: Flexibilidad en Reordenamiento}

Una característica única de NB es permitir reordenar nodos. Al usar orden de Leja en lugar de uniforme, la precisión mejora dramáticamente en casos mal condicionados (ver Tabla 3 del artículo original).

\textit{Conclusión:} Los algoritmos tienen igual complejidad y estabilidad. NB es preferible por:
\begin{enumerate}
\item Derivación transparente (11 líneas vs múltiples técnicas en MM)
\item Facilita reordenamiento de nodos
\item Base teórica más accesible para la comunidad científica no especializada
\end{enumerate}
